\documentclass[en, 11pt]{aghdpl}
\usepackage{amsmath}
\usepackage{epigraph}
\usepackage[english]{babel}
\usepackage[utf8]{inputenc}
\usepackage{enumerate}
\usepackage{listings}
\usepackage{tikz}
\usepackage{amssymb}
\usetikzlibrary{arrows, positioning, calc, backgrounds}
\usepackage{nomencl}
\makenomenclature
\lstloadlanguages{TeX}

\DeclareMathOperator*{\argmax}{arg\, max}
\newcommand\abs[1]{\left|#1\right|}

\author{Sebastian Dziadzio}
\shortauthor{S. Dziadzio}
\titlePL{Zastosowanie morfosyntaktycznych i semantycznych modeli językowych w automatycznym rozpoznawaniu mowy}
\titleEN{Application of Morphosyntactic and~Semantic Language Models in~Automatic Speech Recognition}
\shorttitlePL{Modele językowe w automatycznym rozpoznawaniu mowy}
\shorttitleEN{Language models in Automatic Speech Recognition}

\thesistype{Master of Science Thesis}
\supervisor{dr Bartosz Ziółko}
\date{2015}
\department{Department of Applied Computer Science}
\faculty{Faculty of Electrical Engineering, Automatics, Computer Science and Biomedical Engineering}
\degreeprogramme{Computer Science}

\setlength{\cftsecnumwidth}{10mm}
\setlength{\skip\footins}{1cm}
\setlength{\footskip}{2cm}
\setcounter{secnumdepth}{4}

\begin{document}
\titlepages

\newpage
\thispagestyle{empty}
\begin{flushright}
\null
\vfill
\textit{The limits of my language means the limits of my world.}

\medskip
\textsc{--- Ludwig Wittgenstein}
\vspace{2\baselineskip}


\textit{I know all those words, but that sentence makes no sense to me}

\medskip
\textsc{--- Matt Groening}
\vspace{2\baselineskip}
\end{flushright}
\newpage

\printnomenclature
\tableofcontents
\chapter{Introduction}
\label{chapter:intro}

\section{Automatic speech recognition}
\label{section:asr}
Automatic speech recognition (ASR) \nomenclature{ASR}{Automatic Speech Recognition} can be defined as independent, computer-driven transcription of spoken language into readable text in real time \cite{stuckless1994developments, jelinek1997statistical}. Although this process can be performed almost effortlessly by the human brain, it is extremely difficult to reverse-engineer\footnote{The observation that low-level sensorimotor skills require far more computational resources than high-level reasoning is known as the Moravec's paradox and has been formulated independently by several artificial intelligence researchers in the 1980s. As Moravec wrote: ``it is comparatively easy to make computers exhibit adult level performance on intelligence tests or playing checkers, and difficult or impossible to give them the skills of a one-year-old when it comes to perception and mobility''\cite{moravec1988mind}.}. Large-vocabulary continuous speech recognition \nomenclature{LVCSR}{Large Vocabulary Continuous Speech Recognition}(LVCSR) falls into two distinct categories: speech transcription and speech understanding. The former aims to find the exact orthographic transcription of analysed utterance, while the latter aims to find its meaning. In this thesis we focus on speech transcription, because its performance can be reliably measured in terms of word recognition errors.

\subsection{Bayesian framework}
In general, the recogniser tries to determine the most likely word sequence $\hat{W}$ from possible hypotheses $W$ given a sequence of observed acoustic features $A$:
\begin{equation}
\label{equation:recogniser}
  \hat{W}=\max_{W}P(W|A)
\end{equation}
We can rearrange the conditional probability using the Bayes rule:
\begin{equation}
  \label{equation:bayesian}
  P(W|A)=\frac{P(A|W)P(W)}{P(A)}\propto{P(A|W)P(W)}
  \nomenclature{$P(W|A)$}{Probability of a word sequence $W$ being produced from acoustic evidence $A$}
\end{equation}
$P(W)$ is estimated by the language model (see section \ref{section:lm}), while $P(A|W)$ is computed using the acoustic model \cite{whittaker2000statistical}. Typical ASR systems use Hidden Markov Models (HMMs) \nomenclature{HMMs}{Hidden Markov Models} to model the sequential structure of speech signal \cite{juang1985mixture, baker1975dragon} and Gaussian Mixture Models (GMMs) \nomenclature{GMMs}{Gaussian Mixture Models} for modelling the emission distribution of HMMs \cite{mohamed2012acoustic, bourlard1994connectionist}.

\subsection{Feature extraction}
Speech is a non-stationary process, so to represent it as a succession of discrete stationary states, it is assumed that its statistical properties are constant over a short period of time. Under this assumption it is possible to extract statistically meaningful acoustic parameters (feature vectors) from a sampled speech waveform. The most popular method of spectral analysis is Linear Predictive Coding \nomenclature{LPC}{Linear Predictive Coding} and Mel-Frequency Cepstral Coefficients \nomenclature{MFCCs}{Mel-Frequency Cepstral Coefficients}. 
\begin{figure}[!ht]
  \label{figure:mfcc}
  \centering
    \begin{tikzpicture}[block/.style = {draw, rectangle, minimum width = 4cm, minimum height = 1cm, font = \small}]
      \node [block] (preemphasis) {Preemphasis};nn
      \node [block, right = of preemphasis] (windowing) {Windowing};
      \node [block, right = of windowing] (dft) {DFT};
      \node [block, below = of dft] (melfilter) {Mel Filters};
      \node [block, below = of melfilter] (log) {Logarithm};
      \node [block, left = of log] (dct) {DCT};
      \node [block, left = of dct] (delta) {Delta Coefficients};
      \path[draw,->]
      (preemphasis) edge (windowing)
      (windowing) edge (dft)
      (dft) edge (melfilter)
      (melfilter) edge (log)
      (log) edge (dct)
      (dct) edge (delta);
    \end{tikzpicture}
  \caption{Calculating MFCC}
\end{figure}

The algorithm for calculating MFCCs is shown in figure \ref{figure:mfcc}. First, the speech waveform is subjected to high-frequency preemphasis in order to compensate for lip radiation and attenuation of high frequencies caused by the sampling process \cite{singh2012preprocessing}. Typically, the signal is passed through a high-pass finite impulse response (FIR) \nomenclature{FIR}{Finite Impulse Response} filter:
\begin{equation}
H(z)=1-\frac{a}{z}
\end{equation}
where $0.9 \leq a \leq 1.0$ (for a different approach see: \cite{nossair1995signal}). The signal is then divided into a sequence of frames using 20-30 ms windows with about 50\% overlap. The extraction takes place by multiplying the value of the signal at time $n$ ($s[n]$) with the value of the window at time $n$ ($w[n]$):
\begin{equation}
  y[n]=s[n]w[n]
\end{equation}
Hamming window is used to avoid discontinuities at boundaries:
\begin{equation}
\label{equation:hamming}
  w[n]=
  \begin{cases}
    0.54-0.46\cos(\frac{2 \pi n}{L}) & 0 \leq n \leq L-1 \\
    0                               & \text{otherwise}
  \end{cases}
\end{equation}

The next step is to extract spectral information from the windowed signal using Discrete Fourier Transform (DFT) \nomenclature{DFT}{Discrete Fourier Transform}. The result of Fourier analysis is the information about the amounts of energy in $N$ evenly-spaced discrete frequency bands:
\begin{equation}
  X[k]=\sum_{n=0}^{N-1}x[n]e^{\frac{-2\pi ikn}{N}}
\end{equation}
However, human hearing is not equally sensitive at all frequency bands. Many studies confirm that above about 500 Hz increasingly large intervals are judged by listeners to produce equal pitch increments \cite{stevens1937scale, fletcher1938loudness}. It was shown that simulating this property of human brain during feature extraction improves ASR performance. The signal is therefora passed through a bank of triangular filters spaced linearly below 1000 Hz and logarithmically above. This corresponds to the mapping between raw acoustic frequency $f$ and mel frequency $m$ \cite{muda2010voice}:
\begin{equation}
  m=1127\ln(1+\frac{f}{700})
\end{equation}

\begin{figure}[!ht]
  \centering
  \includegraphics[width=\textwidth]{img/mel.png}
  \caption{Plot of pitch mel scale versus Hertz scale \cite{vedala2008mel}.}
  \label{figure:mel}
\end{figure}

Another property of human hearing is that we are less sensitive to slight variation in signal level at high amplitudes than at low amplitudes. To model that logarithmic response, we take the log of each of the mel spectrum values.

Finally, cepstrum of the signal is calculated. Formally, it can be defined as the inverse DFT of the log magnitude of the DFT of the signal, although Discrete Cosine Transform (DCT) \nomenclature{DCT}{Discrete Cosine Transform} is also frequently used instead of inverse DFT:
\begin{equation}
  \label{equation:cepstrum}
  c[n]=\sum_{n=0}^{N-1}\log{\left(\abs{\sum_{n=0}^{N-1}x[n]e^{\frac{-2\pi ikn}{N}}}\right)}e^{\frac{2\pi ikn}{N}}
\end{equation}

In general, speech can be represented with the source-filter model. The original glottal source waveform of particular fundamental frequency is passed trough the vocal tract, which acts as a filter. However, glottal source features (fundamental frequency $F_{0}$ \nomenclature{$F_{0'}$}{Fundamental frequency}, pulse characteristic etc) are irrelevant for distinguishing phones. Cepstral analysis enables to deconvolve the source from the filter and extract the vocal tract properties, which carry most information about the phone being produced. Higher values on the cepstrum x-axis represent the glottal pulse, while lower values correspond to vocal tract characteristic.Generally, MFCCs are formed from the first 12 cepstral values, which carry information solely about the vocal tract  \cite{jurafsky2000speech}. Another useful property of the cepstral coefficients is that the variance of different coefficients tends to be uncorrelated (which is not true in case of raw mel-frequency spectrum). This means that the GMMs don't have to represent the covariance between MFCCs, which significantly reduces the number of parameters (see section \ref{section:acoustic}).

Having 12 cepstral coefficients for each frame is not sufficient, as we need inormation about one more feature that is useful in phone recognition: energy. The energy of a signal $x$ in a window from time sample $t_{1}$ to time sample $t_{2}$ can be calculated as the sum over time of the power of the samples:
\begin{equation}
  \text{Energy}=\sum_{t=t_{1}}^{t_{2}}x^{2}[t]
\end{equation}

We also have to take into account the variability of the signal. Changes from frame to frame, such as the slope of a formant at its transitions, provide a useful cue for phone identity. For each of the 13 features (12 cepstral features and energy), we calculate the delta (velocity) and double delta (acceleration) features. This can be done simply by computing differences between cepstral values (although usually more sophisticated estimates are used):

\begin{equation}
  d(t)=\frac{c(t+1)-c(t-1)}{2}
\end{equation}

After adding energy coefficients and time derivatives, we end up with a total of 39 MFCCs:
\begin{itemize}
\itemsep0em
\item 12 cepstral coefficients
\item 12 delta cepstral coefficients
\item 12 double delta cepstral coefficients
\item 1 energy coefficient
\item 1 delta energy coefficient
\item 1 double delta energy coefficient
\end{itemize}

\subsection{Acoustic model}
\label{section:acoustic}
To estimate the posterior probability $P(A|W)$ in equation \ref{equation:bayesian}, we use an acoustic model, which represents the mapping from feature vector sequences to some linguistic units (usually phoneme-sized). Hidden Markov Models are most commonly used, although other techniques, such as neural networks, segmental models, and conditional random fields, have also been applied succesfully \cite{yu2009hidden, yu2008maximum, mohamed2012acoustic}. In case of HMMs, the training data usually comprises of 

\subsection{Language models}
\label{section:lm}
 In a Bayesian framework presented in equation \ref{equation:bayesian}, the language model estimates the a priori likelihood by assigning probability $P(W)$ to each word sequence $W=\{w_{1}, \ldots, w_{n}\}$ such that $\sum_{W}P(W)=1$ \cite{rosenfeld2000two}. Since the search is usually performed unidirectionally, $P(W)$ can be formulated as a chain rule:
\begin{equation}
  P(W)=\prod^{n}_{i=1}P(w_{i}|h_{i})
  \nomenclature{$P(W)$}{Probability of a word sequence $W=\{w_{1}, \ldots, w_{n}\}$}
  \nomenclature{$P(w_{i}|h_{i})$}{Probability of a word $w_{i}$ given its history $h_{i}$}
\end{equation}
where $h_{i}=\{w_{1}, \ldots, w_{i-1}\}$ is the word history for $w_{i}$, often reduced to equivalence class $\phi(h_{i})$:
\begin{equation}
  P(w_{i}|h_{i})\approx P(w_{i}|\phi(h_{i}))
  \nomenclature{$\phi(h_{i})$}{Equivalence class of word history $h_{i}$}
\end{equation}
Good equivalence classes maximise information about the next word $w_{i}$ given its history, but also require a vast quantity of example sequences. The development of effective statistical language models is therefore limited by the availability of representative and machine readable text corpora.

\section{Polish vs. English}
\label{section:polish}

\section{Outline}
\label{section:outline}

\chapter{Statistical language models}
\label{chapter:lm}

\section{N-gram models}
\label{section:ngrams}
In Section \ref{subsection:lm} we defined the chain rule of probability, which lets us decompose the joint probability of a sequence of $N$ words into a product of conditional probabilities. Let $w_{1}^{n}$ denote $w_{1}, \dots, w_{n}$. The chain rule can be formulated as
\begin{equation}
	\label{equation:chain}
	P(w_{1}^{n})=\prod_{i=1}^{n}P(w_{i}|w_{1}^{i-1})=\prod_{i=1}^{n}P(w_{i}|h_{i}).
\end{equation}
However, there is no way of computing the probability of a word given a long history of preceding words. Estimating it from relative frequency counts is infeasible, if only for data sparsity reason. We can deal with this problem by using an \mbox{$n$-gram} model. An \mbox{$n$-gram} is simply a sequence of $n$ words (or other modelling units)~--~an \mbox{$n$-gram} of size 1 is called a unigram, size 2 is a bigram, and size 3 is called a trigram. The \mbox{$n$-gram} model approximates the probability of a word given all the previous words with the probability of the word given $n-1$ preceding words:
\begin{equation}
	P(w_{i}|w_{1}^{i-1})\approx P(w_{i}|w_{i-n+1}^{i-1}).
\end{equation}
The \mbox{$n$-gram} can be interpreted as a left-to-right Markov chain of order $n-1$, because the probability of the current state depends only on $n-1$ previous states. To put it differently, the \mbox{$n$-gram} models satisfies the $n-1$ order Markov assumption (see Section \ref{subsection:acoustic}). In contrast to \glspl{hmm}, the states of Markov chains are directly visible.

Another way to look at the \mbox{$n$-gram} model is through the notion of history equivalence class, introduced in Section \ref{subsection:lm}. As indicated previously, it is impossible to obtain a probability estimate for every imaginable word history, but the number of parameters can be reduced by classifying word histories into equivalence classes. Indeed, in \cite{jelinek1997statistical}, language modelling is defined as the task of finding the best history classification function:
\begin{equation}
	\Phi:h_{i}\mapsto\varphi_{i}=\Phi(h_{i})=\Phi(w_{1}^{i-1})
\end{equation}
Finding an optimal history classification function is the matter of striking a fine balance between its predictive power and complexity. To be a useful predictor, an equivalence class has to maximise the information about next word, while minimising the dimensionality of the resulting model. In case of \mbox{$n$-gram} models, the history classification function $\Phi$ simply limits the word history to the last $n-1$ elements:
\begin{equation}
	\Phi(w_{1}^{i-1})=w_{i-n+1}^{i-1}.
\end{equation}
Two word histories are therefore equivalent if their last $n-1$ elements are identical. The tradeoff between efficiency and complexity quickly becomes apparent when the value of $n$ is \mbox{increased~--~higher} order \mbox{$n$-gram}s typically exhibit lower perplexity (see Section \ref{subsection:perplexity}) and better performance, but are more expensive in terms of time and memory.
\subsection{Maximum Likelihood Estimation}
\label{subsection:mle}
The most common technique of estimating the \mbox{$n$-gram} probabilities from a text corpus is \gls{mle}. The procedure is very simple~--~the \mbox{$n$-gram} probability of a word $w_{i}$ given the truncated history $\varphi_{i}=w_{i-n+1}^{i-1}$ is the observed frequency of the \mbox{$n$-gram} $w_{i-n+1}^{i}$ normalised by the sum of observed frequencies of all \mbox{$n$-gram}s sharing the same history:
\begin{equation}
	\label{equation:mle}
	P(w_{i}|\varphi)=\frac{C(\varphi, w_{i})}{\sum_{w_{j}}C(\varphi, w_{j})}.
\end{equation}
Note that the sum in the denominator is equal simply to $C(\varphi_{i})$, so the formula in Equation \ref{equation:mle} can be more intuitively understood as the proportion of cases in which a particular equivalent word history $\varphi_{i}$ is followed by the word $w_{i}$:
\begin{equation}
	\label{equation:mle2}
	P(w_{i}|\varphi_{i})=\frac{C(\varphi_{i}, w_{i})}{C(\varphi_{i})}.
\end{equation}
This ratio is called the relative frequency. In the special case of unigram probability it can be calculated as the count of the particular word $w_{i}$ normalised by the size of the corpus:
\begin{equation}
	\label{equation:mle2}
	P(w_{i})=\frac{C(w_{i})}{\sum_{w}C(w)}.
\end{equation}
\begin{table}[h!]
	\caption[Calculating selected bigram and trigram probabilities using maximum likelihood estimation]{Calculating selected bigram and trigram probabilities using maximum likelihood estimation \cite{jurafsky2000speech}. The \texttt{<s>} and \texttt{</s>} tags denote the beginning and the end of the sentence.}
	\label{table:mle}
	\texttt{%
		\begin{tabular}{c}
			<s> egg bacon and spam </s> \\
			<s> egg bacon sausage and spam </s> \\
			<s> spam bacon sausage and spam </s> \\
			<s> spam egg spam spam bacon and spam </s> \\
			\\
		\end{tabular}}
		\centering
		\begin{tabular*}{.8\linewidth}{@{\extracolsep{\fill}}lll}
			$P(\texttt{bacon}|\texttt{spam})=\frac{2}{8}$ & $P(\texttt{spam}|\texttt{spam})=\frac{1}{8}$ & $P(\texttt{egg}|\texttt{<s> <s>})=\frac{2}{4}$  \\
			$P(\texttt{bacon}|\texttt{egg})=\frac{2}{3}$  & $P(\texttt{spam}|\texttt{<s>})=\frac{2}{4}$ & $P(\texttt{egg}|\texttt{bacon and})=0$ \\
			$P(\texttt{bacon}|\texttt{and})=0$            & $P(\texttt{spam}|\texttt{and})=1$ & $P(\texttt{</s>}|\texttt{and spam})=1$ \\
		\end{tabular*}
	\end{table}
Table \ref{table:mle} shows an example of calculating \mbox{$n$-gram} probabilities using a corpus of four phrases from Monty Python's notorious spam sketch. Inspecting it reveals a major problem with the \gls{mle} approach. Note that some probability estimates are equal to one, so using the model to generate random sentences would in many cases result in phrases taken verbatim from the corpus. Even worse, there are also probability estimates equal to zero. This means that because of the chain rule, the model will asign a zero probability to any sequence containing a known word in a new context. This includes one of the previous lines of the sketch:
	\begin{equation}
		P(\texttt{bacon}|\texttt{and})=0 \Rightarrow P(\texttt{<s> egg and bacon </s>})=0
	\end{equation}
	Both these artifacts are a consequence of data sparsity~--~the corpus is simply too small for the model to be an accurate representation of the language. The probability of observed items is overestimated, while the probability of unobserved items is underestimated. Although the example is obviously exaggerated, the problem persists in case of full-sized corpora. There are many techniques of correcting this bias by shifting the probability mass from frequent to previously unseen items. This process, called smoothing or discounting, is described in Section \ref{subsection:smoothing}. 

	Zero probability is one problem, but there is also an issue of words that do not appear in the corpus at all. In speech recognition, \gls{oov} tokens are inevitable and they need to be identified, because they contribute to recognition errors in surrounding words. More importantly, they are often information-rich nouns, such as proper names, domain-specific notions, or foreign words. Language models can be used to facilitate the process of \gls{oov} token detection by incorporating the information about unknown words into the training process. All words that appear in the \gls{lm} training data, but do not appear in the \gls{asr} vocabulary, are simply substituted by the unknown word token \texttt{<ign>}. These pseudo-words are then treated like any other regular word in the corpus, similarly to \texttt{<s>} and \texttt{</s>}. 

	Another conclusion that can be drawn from the example in Table \ref{table:mle} is that the model is only as representative as the corpus it is trained on. This is especially important in \gls{lvcsr}, where the model not only has to represent non-domain-specific vocabulary, but also focus on spoken language, which is often very different from writing. \gls{asr} systems are often trained on written texts, because this kind of data is usually readily available, but using speech transcripts can lead to better results with less training data \cite{dziadzio2015comparison}. This idea is further explored in the experimental part of the thesis.
	\subsection{Smoothing}
	\label{subsection:smoothing}
	Smoothing is a way of dealing with the zero probability problem. In principle, it is the process of adjusting the \gls{mle} estimates to produce more accurate probabilities \cite{chen1996empirical}. High probabilities are adjusted downwards and low probabilities are adjusted upwards, so the resulting distribution is more uniform. The simplest method of smoothing is additive smoothing. The idea is to add a constant $\delta$ to every \mbox{$n$-gram} count:
	\begin{equation}
		P_{\textsc{add}}(w_{i}|\varphi{i})=\frac{\delta+C(\varphi_{i}w_{i})}{\delta|V|+C(\varphi_{i})},
	\end{equation}
where typically $0 < \delta \leq 1$. This method is easy to implement, but has been shown in \cite{Gale94what'swrong} to generally perform poorly. That is why Katz back-off and Chen and Goodman's modified Knesser-Ney smoothing are used in the experimental part.
	\subsubsection*{Katz smoothing}
	Katz smoothing builds on the Good-Turing estimate, based on the \textit{symmetry requirement} which states that two events which occur the same number of times in the sample must have equal probabilities \cite{whittaker2000statistical}. The idea is to adjust the count of \mbox{$n$-grams} that occur $c$ times using the counts of \mbox{$n$-grams} that occur $c+1$ times~--~in particular, to estimate the probability of unseen \mbox{$n$-grams} using the singleton counts. Let $n_{c}$ denote the number of \mbox{$n$-grams} that occur $c$ times in the sample. For each count $c$, an adjusted count $\hat{c}$ is computed:
	\begin{equation}
		\hat{c}=(c+1)\frac{n_{c+1}}{n_{c}}.
		\label{equation:gt}
	\end{equation}
	From Equation \ref{equation:gt} it follows that the total number of counts that will be assigned to \mbox{$n$-grams} with zero counts is equal to the number of singletons. The updated probability estimate can be calculated by normalising the updated count by the total number of tokens. For an \mbox{$n$-gram} $\alpha$ with $c_{\alpha}$ counts:
	\begin{equation}
		P_{\textsc{good}}(\alpha:C(\alpha)=c_{\alpha})=\frac{\hat{c_{\alpha}}}{\sum_{c}cn_{c}}.
	\end{equation}

	The problem with using plain Good-Turing estimates for discounting is that $n_{c+1}$ quite often equals zero for high $c$. Katz smoothing extends the idea behind Good-Turing estimation by using lower-order distributions to better reallocate the count mass subtracted from nonzero counts. For example, while adjusting the counts of bigrams, the unigram distribution is used: 
	\begin{equation}
		C_{\textsc{katz}}(w_{i-1}, w_{i})=
		\begin{cases}
			d_{c}c & \text{if } c>0 \\
			\alpha(w_{i-1})P_{\textsc{mle}}(w_{i}) & \text{if } c=0,
		\end{cases}
	\end{equation}
	The discount coefficient $d_{c}$ depends on the the \mbox{$n$-gram} count $c$. Counts larger than some arbitrary treshold $k$ are not discounted ($d_{c}=1$). The value of $d_{c}$ when $c\leq k$ is chosen so that the resulting discount is proportional to the Good-Turing discount: 
	\begin{equation}
		1-d_{c}=\mu(1-\frac{\hat{c}}{c}).
		\label{equation:constraint1}
	\end{equation}
	Furthermore, the total number of counts discounted in the \mbox{$n$-gram} distribution should be equal to the total number of counts assigned to zero-count \mbox{$n$-grams} according to the Good Turing estimate (see Equation \ref{equation:gt}):
	\begin{equation}
		\sum_{c=1}^{k}n_{c}(1-d_{c})c=n_{1}.
		\label{equation:constraint2}
	\end{equation}
	The only solution that satisfies the constraints formulated in Equation \ref{equation:constraint1} and \ref{equation:constraint2} is given by:
	\begin{equation}
		d_{c}=\frac{\frac{\hat{c}}{c}-\frac{(k+1)n_{k+1}}{n_{1}}}{1-\frac{(k+1)n_{k+1}}{n_{1}}}.
		\label{equation:dc}
	\end{equation}
	Katz smoothing for higher-order \mbox{$n$-gram} is defined recursively, where the unigram model is taken to be the \gls{mle} unigram model to end the recursion \cite{whittaker2000statistical}. The updated probability $P_{\textsc{katz}}$ of word~$w_{i}$ given truncated history $\varphi=w_{i-n+1}^{i-1}$ is given by:
	\begin{equation}
		P_{\textsc{katz}}(w_{i}|\varphi)=
		\begin{cases}
			\frac{C(\varphi, w_{i})}{C(\varphi)}& \text{if } C(\varphi, w_{i}) > k\\
			d_{C(\varphi, w_{i})}\cdot \frac{C(\varphi, w_{i})}{C(\varphi)}& \text{if } 1 \leq C(\varphi, w_{i}) \leq k\\
			\alpha(\varphi)\cdot P_{\textsc{katz}}(w_{i}|w_{i-n+2}^{i-1})& \text{if } C(\varphi, w_{i})=0,
		\end{cases}
	\end{equation}
	where:
	\begin{equation}
		\alpha(\varphi)=\frac{1-\sum_{w_{i}:C(\varphi, w_{i})>0}P_{\textsc{katz}}(w_{i}|\varphi)}{\sum_{w_{i}:C(\varphi, w_{i})=0}P_{\textsc{katz}}(w_{i}|w_{i-n+2}^{i})}.
	\end{equation}
        For details on the derivation of the back-off weight~$\alpha$ and the discount coefficient~$d_{c}$, see \cite{chen1996empirical}.
	\subsubsection*{Kneser-Ney smoothing}
	The Kneser-Ney smoothing is an extension of absolute discounting, which involves subtracting a fixed discount $\delta \in (0, 1)$ from each nonzero count:
	\begin{equation}
		P_{\textsc{abs}}(w_{i}|w_{i-n+1}^{i-1})= \frac{max\{C(w_{i-n+1}^{i})-\delta, 0\}}{\sum_{w_{i}}C(w_{i-n+1}^{i})}+(1-\lambda_{w_{i-n+1}^{i}})P_{\textsc{abs}}(w_{i}|w_{i-n+2}^{i-1}).
	\end{equation}
	The parameter $\lambda$ is taken so that the resulting distribution sums to one:
	\begin{equation}
		1-\lambda_{w_{i-n+1}^{i-1}}=\frac{\delta}{\sum_{w_{i}}C(w_{i-n+1}^{i})}N_{1+}(w_{i-n+1}^{i-1}~\bullet).
	\end{equation}
	The expression $N_{1+}(\varphi~\bullet)$, denotes the number of unique words that follow the history $\varphi$:
	\begin{equation}
		N_{1+}(w_{i-n+1}^{i-1}~\bullet)=|\{w_{i}:C(w_{i-n+1}^{i}) > 0\}.
	\end{equation}
	The parameter $\delta$ is chosen using held-out estimation. In \cite{ney1994structuring}, it is estimated as 
	\begin{equation}
		\delta=\frac{n_{1}}{n_{1}+2n_{2}}.
	\end{equation}
	The idea behind Kneser-Ney smoothing is that since the lower-order model is only necessary when the count is small or zero in the higher-order model, it should be optimised for that purpose. In most other algorithms, the lower-order distribution is just a smoothed version of the MLE distribution. In Kneser-Ney smoothing, the unigram probability is not proportional to the number of occurences of the word, but the number of words it follows:
	\begin{equation}
		P_{\textsc{kn}}(w_{i})=\frac{N_{1+}(\bullet~w_{i})}{N_{1+}(\bullet~\bullet)}
		\label{equation:kneser-ney}
	\end{equation}
	where $N_{1+}(\bullet~w_{i})$ is the number of different words that precede $w_{i}$ in the training data and $N_{1+}(\bullet~\bullet)$ is the number of unique bigrams with nonzero count:
	\begin{equation}
		N_{1+}(\bullet~w_{i})=|{w_{i-1}:C(w_{i-1}, w_{i})>0}|
	\end{equation}
	\begin{equation}
		N_{1+}(\bullet~\bullet)=|{(w_{i-1}, w_{i}):C(w_{i-1}, w_{i})>0}|
	\end{equation}
	The general formula for the unmodified Kneser-Ney smoothing is
	\begin{equation}
		P_{\textsc{kn}}(w_{i}|w_{i-n+1}^{i-1})= \frac{max\{C(w_{i-n+1}^{i})-\delta, 0\}}{\sum_{w_{i}}C(w_{i-n+1}^{i})}+(1-\lambda_{w_{i-n+1}^{i}})\frac{N_{1+}(\bullet~w_{i-n+2}^{i})}{N_{1+}(\bullet~w_{i-n+2}^{i-1}~\bullet)},
	\end{equation}
	where:
	\begin{equation}
		N_{1+}(\bullet~w_{i-n+2}^{i})=|{w_{i-n+1}:C(w_{i-n+1}^{i})>0}|,
	\end{equation}
	\begin{equation}
		N_{1+}(\bullet~w_{i-n+2}^{i-1}~\bullet)=|{w_{i-n+1, w_{i}}:C(w_{i-n+1}^{i})>0}|.
	\end{equation}
	The Chen and Goodman's modification of the original Kneser-Ney algorithm uses three different discount values, $\delta_{1}$, $\delta_{2}$, $\delta_{3}$, that are applied to \mbox{$n$-gram}s with one, two, and three or more counts, respectively \cite{chen1996empirical}. While estimating $P_{\textsc{kn}}(w_{i}|w_{i-n+1}^{i-1})$, the parameter $\delta$ simply becomes a function of $C(w_{i-n+1}^{i-1})$: 
	\begin{equation}
		D(c)=	
		\begin{cases}
			0 & \text{if } c=0\\
			\delta_{1} & \text{if } c=1\\
			\delta_{2} & \text{if } c=2\\
			\delta_{3+} & \text{if } c\geq3\\
		\end{cases}
	\end{equation}
	The distribution must still sum to one, so the new value of $\alpha$ is
	\begin{equation}
		\alpha(w_{i-n+1}^{i-1})=\frac{\delta_{1}N_{1}(w_{i-n+1}^{i-1}~\bullet)+\delta_{2}N_{2}(w_{i-n+1}^{i-1}~\bullet)+\delta_{3+}N_{3+}(w_{i-n+1}^{i-1}~\bullet)}{\sum_{w_{i}}C(w_{i-n+1}^{i})}
	\end{equation}
	The modified version has been shown to significantly outperform the original Kneser-Ney algorithm, because the optimal average discount for \mbox{$n$-gram}s with one or two counts is different from the optimal average discount for \mbox{$n$-gram}s with higher counts. Chen and Goodman provide estimates of the optimal values for $\delta_{1}$, $\delta_{2}$, and $\delta_{3}$:
	\begin{align}
		Y&=\frac{n_{1}}{n_{1}+2n_{2}} \nonumber\\ 
		\delta_{1}&=1-2Y\frac{n_{2}}{n_{1}} \nonumber\\
		\delta_{2}&=2-3Y\frac{n_{3}}{n_{2}} \nonumber\\
		\delta_{3+}&=3-4Y\frac{n_{4}}{n_{3}}.
	\end{align}
	\subsection{Word n-grams}
	So far, while describing the \mbox{$n$-gram} models, we used the terms ``word'' and ``modeling unit'' interchangeably. However, depending on the application, \mbox{$n$-gram}s can use sub-lexical units, such as phonemes, letters, and syllables, or supra-lexical units, such as \gls{pos} tags. In case of \gls{asr}, words are the most popular choice for the modeling units, because word-based models can be easily incorporated into the search process described in section \ref{subsection:lm}. For example, consider this famous quote:
	\begin{center}
		\texttt{open the pod bay doors}
	\end{center}
	We would like to estimate the probability of next word being \texttt{hal}. Using a bigram model, we get:
	\begin{equation}
		P(\texttt{hal}|\texttt{open the pod bay doors}) \approx P(\texttt{hal}|\texttt{doors})
	\end{equation}
	Truncating the history to $n-1$ units drastically reduces the number of free parameters. Although this number is still enormous - for a bigram model and a vocabulary of 50000 words, there are potentially two and a half billion parameters~--~it is at least feasible to estimate the probability of the entire sequence using the chain rule from Equation \ref{equation:chain}:
	\begin{multline}
		P(\texttt{<s> open the pod bay door hal </s>}) \approx \\
		\approx P(\texttt{open}|\texttt{<s>})P(\texttt{the}|\texttt{open})P(\texttt{pod}|\texttt{the})P(\texttt{bay}|\texttt{pod})P(\texttt{door}|\texttt{bay})P(\texttt{hal}|\texttt{door})P(\texttt{</s>}|\texttt{hal})
	\end{multline}
	There are several disadvantages of word-based \mbox{$n$-gram} models. The first one is the obviously unrealistic Markov's assumption. Language dependencies often span across far more than just three or four words, especially in inflected languages. Consequently, low-order word-based \mbox{$n$-gram}s may do a poor job at disambiguating grammatically invalid sentences, because an incorrect sentence can be constructed from a sequence of valid short \mbox{$n$-gram}s. Take this quote from Yoda as an example:
	\begin{center}
		\texttt{<s> look I so old to young eyes </s>}
	\end{center}
	A bigram model would likely assign a relatively high probability to this phrase, because each pair of consecutive words is entirely plausible. 
	In the above example, using a trigram model would probably yield better results, but at a considerable cost - even for a lexicon of 50 000 words (far too few for modeling any inflected language), switching from bigram to trigram introduces more than one hundred trillions potential parameters! In short, there is always a tradeof between the quality of predictions and the amount of data required to calculate the probability estimates. Training and storing higher-order word \mbox{$n$-gram}s can be simply infeasible due to data sparsity and memory constraints.
	Another issue with word-based \mbox{$n$-gram}s is that the truncation of word history may lead to unintuitive or inconsistent behaviour. This problem is especially pronounced in case of languages with weak constraints on word order~--~permutations of the same sequence may be assigned different probability estimates despite being grammatically and semantically equivalent. Furthermore, very similar word histories may lead to different predictions. Consider these two sequences:
	\begin{center}
		\texttt{a brilliant american mathematician who graduated from} \\
		\texttt{a brilliant american mathematician who graduated recently from}.
	\end{center}
	From the point of view of predicting the next word, these two histories are very similar. However, they would be considered completely different by a trigram word model \cite{whittaker2000statistical}. In the second case, we would need at least a $5$-gram model to capture the intuition that the next word is probably a name of a university.

	Although word-based \mbox{$n$-gram} language models have been outperformed by deep neural networks in terms of perplexity and word error reduction rate, they are still commonly used in \gls{asr} systems, as they strike a fine balance between effectiveness and simplicity. They are straightforward to train and can be easily used to guide the search through the lattice of word hypotheses. However, in case of inflected languages, the problem of data sparsity is significantly amplified and therefore using words as the modeling unit is not necesarily the best choice.
	\subsection{Class-based \mbox{$n$-gram} models}
	\label{subsection:class}
	Class-based models address the problem of data sparsity by reducing the number of parameters. The idea is to cluster words with similar statistical distributions or linguistic properties into groups. A class \mbox{$n$-gram} model with $C$ classes and a vocabulary of size $V$ has $V-C$ word emission parameters and $C^{n}-1$ independent \mbox{$n$-gram} parameters \cite{brown1992class}. Class-based models always have fewer parameters than analogous word-based models, and therefore their parameters can be more accurately estimated, as most classes will be well represented in the corpus. Class \mbox{$n$-gram} models are constructed in a similar fashion as word-based models, except that words are mapped to equivalence classes: 
	\begin{equation}
		\pi:w_{i}\mapsto \gls{mapping}
		\label{equation:deterministic_class}
	\end{equation}
	The class mapping $\pi$ can be deterministic or probabilistic. An example of a deterministic mapping is the \texttt{<ign>} token described in Section \ref{subsection:mle}~--~a word either appears in the vocabulary or not, so the mapping is unambiguous. In contrast, clustering based on linguistic knowledge is often ambiguous, because the same word can belong to different grammatical categories, depending on the context. 
	In case of a deterministic mapping, the probability of a word given its history can be calculated as the product of the probability of a particular word given its class (word emission probability) and the probability of a certain class given a history of $n-1$ classes:
	\begin{equation}
		P_{\textsc{class}}(w_{i}|w_{i-n+1}^{i-1})=P(w_{i}|\pi(w_{i}))P(\pi(w_{i})|\pi(w_{i-n+1}), \dots, \pi(w_{i-1}))
		\label{equation:classngram}
	\end{equation}
	The class \mbox{$n$-gram} probabilities can be calculated from the corpus using \gls{mle}, similarly to word \mbox{$n$-gram} probabilities. Note that a history equivalence class can be used in the The word emission probability is often dropped, but can be otherwise estimated as the relative frequency of the form:
	\begin{equation}
		P(w_{i}|c_{i})=\frac{C(w_{i})}{C(\pi(w_{i}))}
		\label{equation:emission_probability}
	\end{equation}
	In case of a probabilistic mapping, there are multiple realisations of the same word history, as each word can belong to several classes. Therefore, the prediction of the current word requires a summation over the probabilities of all the posible realisations \cite{ney1994structuring}.
	The methods for finding the class mapping function fall roughly into two categories~--~knowledge-based and data-driven. The former approach takes advantage of prior linguistic knowledge. Examples include clustering based on \gls{pos} tags or semantic functions. This method usually requires additional resources such as tagged corpora, wordnets, or automatic morphological taggers. The data-driven clustering generally uses a greedy algorithm to automatically cluster words in such a way that the perplexity of the corpus is minimised. The class-based models used in this work are deterministic, knowledge-based morphosyntactic models.
	\section{Neural network models}
	\label{section:rnn}
	For decades, n-grams have been the standard approach to statistical language modelling. Although many alternative techniques were proposed, the improvements in performance usually came at the cost of computational complexity \cite{mikolov2011rnnlm}. The neural network based language models were introduced in \cite{bengio2003neural} and motivated by the observation that the word-based n-gram models have two major setbacks~--~they are unable to capture longer contexts and they ignore the similarity between words. 
	\subsection{The feedforward architecture}	
The neural models use a distributed representation to deal with the high dimensionality of the training data. The idea is to represent each word as a real-valued vector in $\mathbb{R}^{m}$ and then express the probability function of word sequences in terms of these vectors. The probability function are smooth functions of the word representation, so these kind of models generalize much better to unknown sequences. Because the word embedding $W$: $words \mapsto \mathbb{R}^{m}$ has a useful property of clustering together synonyms and words from the same class, it allows to deal with the data sparsity by generalizing every training sentence to a class of similar sentences. The embedding and the parameters of the probability function can be learned simultaneously. In \cite{bengio2003neural}, a feedforward multi-layer neural network implementing this framework was shown to yield a much better perplexity than the Kneser-Ney backoff n-gram models, but at the cost of high computational complexity and complex implementation.
	\subsection{The recurrent architecture}
	In \cite{mikolov2011extensions}, a recurrent network architecture was shown to outperform the feedforward one. The \glspl{rnn} have a simple implementation and a very useful ability to store information in the hidden layer. The architecture of neural networks used in the experiments is described in \cite{mikolov2010recurrent} and presented in Figure \ref{figure:recurrent}. The input layer is formed by concatenating $w(t)$, a vector representing the current word with $s(t-1)$, the output of the hidden layer from the previous time step:
	\begin{equation}
		x(t)=[w(t)^{T}s(t-1)^{T}]^{T}
		\label{equation:word_vector}
	\end{equation}

	\begin{figure}[htbp]
		\centering
		\includesvg{rnn}
		\caption{A simple \gls{rnn} called an Elman network, based on \cite{mikolov2011extensions}}
		\label{figure:recurrent}
	\end{figure}
	The word vector $w(t)$ has the same size as the vocabulary. It uses 1-of-$N$ coding, meaning that the $i$-th word of the vocabulary is coded by setting the $i$-th element to one and all the other elements to zero \cite{schwenk2005training}. The output layer $y(t)$ has the same dimensionality and represents the probability distribution of the next word. The network is trained using the standard backpropagation algorithm, described in \cite{rumelhart1988learning}. The output of the hidden layer is computed by applying the sigmoid activation function \gls{sigmoid} on the product of the input and the weight matrix $U$:
	\begin{equation}
		s_{j}(t)=f(\sum_{i}x_{i}(t)u_{ji}),
		\label{equation:hidden}
	\end{equation}
where:
\begin{equation}
	f(z)=\frac{1}{1+e^{-z}}.
	\label{equation:sigmoid}
\end{equation}
The values of the ouput layer are calculated by multiplying the output of the hidden layer by the weight matrix $V$ and applying the softmax function:
\begin{equation}
	y_{k}(t)=g(\sum_{j}s_{j}(t)v_{kj}),
	\label{equation:output}
\end{equation}
The softmax function \gls{softmax} transforms a real-valued vector $z$ into a vector $\sigma$ of real values in the range (0, 1) adding up to one, so that it forms a valid probability distribution:
\begin{equation}
	\sigma(z)_{j}=\frac{e^{z_{j}}}{\sum_{k}e^{z_{k}}}.
	\label{equation:softmax}
\end{equation}

\subsection{Extensions to the RNN model}
The \glspl{rnn} used in the experimental take advantage of two important techniques~--~\gls{bptt} and output layer factorization. The former aims to improve the model performance, while the latter allows to speed up the training process. BPTT is an extension of the backpropagation algorithm on the recurrent networks. The network is unfolded by duplicating the recurrent weight for an arbitrary number of time steps $\tau$, and the error is propagated back through an unfolded network, as presented in Figure \ref{figure:unfolding}.

	\begin{figure}[htbp]
		\centering
		\includesvg{unfold}
		\caption{An unfolded network for BPTT with $\tau$=2, based on \cite{boden2002guide}}
		\label{figure:unfolding}
	\end{figure}

	To speed up the training, one could use a technique analogous to the one described in Section \ref{subsection:class}. Assumming that each word belongs to exactly one class, the probability of the word given its history can be calculated as in equation \ref{equation:classngram}. This reduces the computational complexity from $(1+H)H\tau+HV$ to $(1+H)H\tau+HC$, where $H$ is the size of the hidden layer, $V$ is the size of the vocabulary, $\tau$ is the number of \gls{bptt} steps, and $C$ is the number of classes. That can be a substantial improvement, as the $HV$ term is usually the computational bottleneck and of course $C$ is chosen so that $C\ll V$. However, the neural network architecture allows to extend this idea and assume that the emission probability depends on the hidden layer $s(t)$. Equation \ref{equation:classngram} becomes:
	\begin{equation}
		P_{\textsc{class}}(w_{i}|w_{i-n+1}^{i-1})=P_{0}(w_{i}|\pi(w_{i}), s(t))P_{1}(\pi(w_{i})|s(t))
		\label{equation:classneural}
	\end{equation}
	Words are assigned to classes using the frequency binning with the amount of classes as a parameter. The modified network architecture is presented in Figure \ref{figure:rnnclass}. 

	\begin{figure}[htbp]
		\centering
		\includesvg{rnnclass}
		\caption{RNN with a factorized output layer, based on \cite{mikolov2011extensions}}
		\label{figure:rnnclass}
	\end{figure}
	
	The probability distribution is first estimated over the classes and then over the words from the class containing the predicted word:
	\begin{equation}
		c_{l}=g(\sum_{j}s_{j}(t)w_{lj})
		\label{equation:classoutput}
	\end{equation}
	\begin{equation}
		y_{c}(t)=g(\sum_{j}s_{j}(t)v_{cj})
		\label{equation:wordoutput}
	\end{equation}

 	\section{Evaluation of language models}
	\label{section:evaluation}
	The best way to evaluate a language model in the context of automatic speech recognition is to simply incorporate it in an existing \gls{asr} system and measure the performance. This approach, known as extrinsic evaluation, is the only way to know whether a particular model will actually improve the recognition rates, but it can be time-consuming, as it requires running the entire system multiple times \cite{jurafsky2000speech}. Moreover, the recognition task is a very complex one, so there are a lot of factors that can affect the performance of the model. It is therefore more practical to use an intrinsic evaluation metric, such as perplexity, which enables to quickly and objectively measure the quality of a model in an application-agnostic manner. This section describes the evaluation methods used in the experimental part.
	\subsection{Entropy and perplexity}
	\label{subsection:perplexity}
	Perplexity is the most common intrinsic metric of language model quality. It can be derived from the notion of entropy. In information theory, an information source is defined as a device which emits symbols from a finite set $V$. Entropy of an information source can be intuitively understood as the measure of its unpredictability or information content, expressed in bits. For an information source independently emitting symbols $x$ with probability $P(x)$ it is defined as
	\begin{equation}
		H=-\sum_{x}P(x)\log_{2}P(x).
		\label{equation:entropy}
	\end{equation}
	Entropy is maximised when all emission probabilities are equal, that is when $P(x)=\frac{1}{|V|}$. This means that a source with entropy H contains the same amount of information as a source emitting symbols from a set of size $2^{H}$ with probability $2^{-H}$.

	Language can be treated as an information source producing sequences of modelling units $w_{1}, \dots, w_{n}$ with probability $P(w_{1}, \dots, w_{n})$, where each token $w_{i}$ is taken from a vocabulary $V$. Because the symbols are not emitted independently, the notion of per-word entropy is used:
	\begin{equation}
		H=-\lim_{n\to\infty}\frac{1}{n}\sum_{w_{1}, \dots, w_{n}}P(w_{1}, \dots, w_{n})\log_{2}P(w_{1}, \dots, w_{n}).
		\label{equation:genentropy}
	\end{equation}
	The Shannon-McMillan-Breiman theorem states that in case of a stationary and ergodic process, it is possible to use one long sequence in the equation \ref{equation:genentropy} instead of the summation over all possible sequences \cite{algoet1988sandwich}. The assumption here is that a very long sequence will contain most of the shorter sequences and their frequencies will correspond to their probabilities \cite{jurafsky2000speech}. The per-word entropy can therefore be approximated by:
	\begin{equation}
		\gls{entropy}\approx\lim_{n\to\infty}-\frac{1}{n}\log_{2}P(w_{1}, \dots, w_{n}).
		\label{equation:approxentropy}
	\end{equation}
	However, the exact probability of a word sequence is unknown, so it is more practical to use the notion of cross-entropy, which enables to replace the actual probability $P(w_{1}, \dots, w_{n})$ with the language model probability estimate $\hat{P}(w_{1}, \dots, w_{n})$. The sequences are generated according to the actual probability distribution, but the probability estimates are used for the log summation:
	\begin{equation}
		\gls{crossentropy}=-\lim_{n\to\infty}\frac{1}{n}\sum_{w_{1}, \dots, w_{n}}P(w_{1}, \dots, w_{n})\log_{2}\hat{P}(w_{1}, \dots, w_{n}).
		\label{equation:crossentropy}
	\end{equation}
	Cross-entropy can be approximated similarly to per-word entropy, using the Shannon-McMillan-Breiman theorem:
	\begin{equation}
		\hat{H}=-\frac{1}{n}\log_{2}\hat{P}(w_{1}, \dots, w_{n}).
		\label{equation:entropymodel}
	\end{equation}
	The quantity in equation \ref{equation:entropymodel} can be thought of as the average amount of bits that are required to specify a word. The difference between the entropy $H$ and cross-entropy $\hat{H}$ is a measure of the accuracy of a model. Since the true entropy is always less than or equal to the cross-entropy, a model with a lower cross-entropy is a better representation of the language. Perplexity of the model $\hat{P}$ on a test text $W=w_{1}, \dots, w_{n}$ is formally defined as the exponentiation of the cross-entropy:
	\begin{equation}
		PP=\sqrt[n]{\frac{1}{P(w_{1}, \dots, w_{n})}}=\sqrt[n]{\prod_{i=1}^{n}\frac{1}{P(w_{i}|w_{1}, \dots, w_{i-1})}}
		\label{equation:perplexity}
	\end{equation}
	Perplexity is equivalent to the weighted average branching factor of a language. A model with a perplexity of $2^{H}$ is as confused while predicting the next word as if there were on average $2^{H}$ equally probable words.

	Despite its popularity as a measure of language model quality, perplexity is a rather poor method of estimating the actual performance of a language model in an \gls{asr} task. A model with lower perplexity is, in some sense, a better representation of the language, but there is no guarantee it will translate to high recognition accuracy. The main reason behind this is that perplexity does not take into account the acoustic similarities between words, which are crucial in the speech recognition process. A model able to accurately discriminate between acoustically similar words often performs better, even if its perplexity is relatively high. Some alternatives to perplexity have been proposed, such as low-level language model estimates, adversarial evaluation, artificial lattices, or classification of pseudo-negative sentences, but the word error rate is the only reliable method of measuring model performance in an \gls{asr} system \cite{pauls2012large}\cite{smith2012adversarial}.
	\subsection{WER}
	\label{subsection:wer}
	\gls{wer} is a common performance metric for \gls{asr} or machine translation systems. It can be defined as the length normalised word-level Levenshtein distance. More intuitively, it is the minimal quantity of insertions, deletions, and substitutions of words required to convert a hypothesis phrase into the reference phrase divided by the length of the reference phrase:
	\begin{equation*}
		WER=\frac{S+D+I}{N}
	\end{equation*}
	where $S$ is the number of substitutions, $D$ is the number of deletions, $I$ is the number of insertions and $N$ is the number of words in the reference phrase. All of these numbers are non-negative, so the minimal value of WER is zero precisely when the hypothesis and the reference are identical. However, there is no upper bound, as the number of insertions can theoretically be arbitrary. \gls{wer} is often expressed as a percentage. It is common to assign different weights to insertions, deletions, and substitutions, but for all \gls{wer} values in this paper they were weighted equally. Table \ref{table:werr} illustrates the WER calculation. 

\begin{table}[h!]
	\caption{An example of WER calculation for a reference phrase \textit{let's recognize speech}}
  \label{table:werr}
    \centering
    \begin{tabular*}{.6\linewidth}{@{\extracolsep{\fill}}lrrrr}
	        hypothesis                  & S & D & I & WER\\
		\midrule
		lets reckon a nice beach    & 3 & 2 & 0 & $\frac{5}{3}$\\
		lets wreck a nice speech    & 2 & 2 & 0 & $\frac{4}{3}$\\
		let us wreck a nice beach   & 3 & 3 & 0 & $\frac{6}{3}$\\
		let's recognize             & 0 & 0 & 1 & $\frac{1}{3}$\\
    \end{tabular*}
\end{table}

Word error rate is commonly used as a speech recognition accuracy metric, although it has been shown that at least in some applications, it is not necessarily the best choice. For example, optimising the parameters of the \gls{asr} component of a \gls{st} system by translation metrics has been shown to lead to greater translation quality, despite a decrease in WER \cite{he2011word}. For the purpose of this thesis, the absolute \gls{wer} reduction (WERR) is used as an evaluation score for language models. WERR is defined as a reduction of \gls{wer} before and after applying the model.

	\subsection{N-best list rescoring}
	One way of calculating the \gls{werr} of a language model is the n-best list rescoring framework. An n-best list is simply a list of $n$ hypotheses that are most probable according to the acoustic model, along with their probabilities. The hypotheses are re-ranked by incorporating the language model information \cite{broman2005methods}. For the purpose of this work, the acoustic model probability ($P_{\textsc{am}}$) was combined with the language model probability ($P_{\textsc{lm}}$) using the log-linear interpolation:
	\begin{equation}
		P_{\textsc{re}}(h)\propto P_{\textsc{am}}^{(1-\alpha)}P_{\textsc{lm}}^{\alpha}.
	\end{equation}
	Table \ref{table:nbest} shows an example of n-best list rescoring. The \gls{werr} of the language model can be calculated as the difference between the \gls{wer} before and after rescoring. The \gls{wer} can be calculated as the \gls{wer} of the best hypothesis, as shown in equation \ref{equation:WERR}, or a weighted average of WER over a set of hypothesis, with \gls{am} or \gls{lm} probabilities as weights, but in the context of an \gls{asr} system, the first method is a more natural evaluation metric.

\begin{table}[h!]
  \caption{N-best list rescoring with $\alpha=0.6$}
  \label{table:nbest}
    \centering
    \begin{tabular*}{.6\linewidth}{@{\extracolsep{\fill}}lrrrr}
	    hypothesis & $P_{\textsc{am}}$ & $P_{\textsc{lm}}$ & $P_{\textsc{re}}$ & WER\\
	        \midrule
		lets reckon a nice beach    & 0.5 & 0.2 & 0.32 & $\frac{5}{3}$\\
		lets wreck a nice speech    & 0.4 & 0.1 & 0.22 & $\frac{4}{3}$\\
		let us wreck a nice beach   & 0.4 & 0.3 & 0.34 & $\frac{6}{3}$\\
		let's recognize             & 0.3 & 0.6 & 0.48 & $\frac{1}{3}$\\
    \end{tabular*}
\end{table}

	\begin{align}
		\label{equation:WERR}
		&WER_{1}=WER(\argmax{P_{\textsc{am}}})=\frac{5}{3} \nonumber\\
		&WER_{2}=WER(\argmax{P_{\textsc{re}}})=\frac{1}{3} \nonumber\\
		&WERR =  WER_{1} - WER_{2}\approx133.33\%
	\end{align}

\chapter{Building the models}
\label{chapter:tools}
\section{Tools and resources}
This section describes the resources that served as a source of training data, and the tools that were used for gathering, filtering, and transforming the data, including XML-parsing, POS-tagging, lemmatization, and morphological disambiguation. It also presents two language modelling frameworks that enabled building large-scale n-gram and neural language models on consumer hardware.
\subsection{NKJP}
\label{section:nkjp}
Collecting high quality language data is a difficult task, as large and representative collections of modern texts are generally hard to obtain, if only for copyright reasons. One potential source of modern texts is obviously the Internet. However, web data can be extremely noisy and scraping, cleaning, and normalizing it would be a cumbersome process. It should also be considered that the language of the Internet is different from that used in speech. Indeed, transcripts of spoken language appear to be a more valuable source of training data in \gls{asr} applications, than written texts \cite{dziadzio2015comparison}. Moreover, using raw text data to build POS-based models would require a very accurate and robust morphological tagger. For these reasons, linguistic corpora have become essential in advanced language technology. Fortunately, there exists an extensive, publicly available reference corpus of Polish language.

The \gls{nkjp} is a shared initiative of four institutions: Institute of Computer Science at the Polish Academy of Sciences, Institute of Polish Language at the Polish Academy of Sciences, Polish Scientific Publishers PWN, and the Department of Computational and Corpus Linguistics at the University of Łódź. It has been carried out as a research-development project of the Ministry of Science and Higher Education. The full corpus contains over one and a half billion words. The list of sources for the corpora consists of literature, scientific journals, magazines, daily newspapers, and a variety of Internet texts. Most importantly, it contains transcripts of parliamentary proceedings and conversations. Moreover, the creators of the corpus claim that the conversations were chosen so that they represent both male and female speakers, in various age groups, coming from various regions in Poland \cite{lewandowska2012narodowy}. For that reason alone, \gls{nkjp} seemed like an excellent choice for the source of training data.

Another important feature of the \gls{nkjp} is that every text is accompanied by several layers of annotation. The metadata contain information about the text, such as title, author, and source, or~--~in case of speech transcripts~--~the topic of the conversation, the level of formality, background information about the speakers, and so on \cite{przepiorkowski2009xml}. More importantly, every lexical unit is described by several tags carrying information about its grammatical class and category. This enabled to easily filter the data by rejecting foreign words, incomplete or corrupted segments, punctuation, and non alpha-numeric characters. 

For the purpose of the experimental part of this thesis, the redistributable subcorpus was used as the main source of training data. It consists of all text of the full \gls{nkjp} that are free from intellectual property constraints. The sources include mostly texts of legal documents and transcripts of parliamentary and investigation commision proceedings. For detailed information about the contents of the final training set, see Section \ref{subsection:trainingset}.

\begin{table}[h!]
  \begin{center}
	  \caption{Contents of the redistributable subcorpus of the NKJP.}
	    \label{table:freenkjp}
	    \begin{tabular*}{.6\linewidth}{@{\extracolsep{\fill}}lr}
      source & word count \\
      \cmidrule{1-2}
      books & 67 000\\
      proceedings of investigation commisions & 4 623 000\\
      legal texts & 6 970 000\\
      parliamentary proceedings & 87 621 000\\
    \end{tabular*}
  \end{center}
\end{table}

\subsection{Concraft-pl}
Concraft-pl is a morphosyntactic tagger for Polish. It combines Maca -- a morphosyntactic segmentation and analysis tool and Concraft -- a morphosyntactic disambiguation library based on constrained conditional random fields \cite{waszczuk2012harnessing}. It was trained on the manually annotated subcorpus of the \gls{nkjp}, so that the tagsets are compatible. The tagger was used to tag plain text data in cases where the manual or semi-automatic annotation was unavailable. Although the output of the tagger is a list of most probable morphosyntactic tags with corresponding probabilities, only the one chosen by the disambiguation library was taken into consideration. The class mapping is therefore deterministic, which simplifies the languague model and enables to use existing \gls{nkjp} annotation when possible. As for lemmatization, Concraft-pl doesn't disambiguate between lemmas when the morphosyntactic tag is the same, so the NKJP annotation was used in almost every case and a random hypothesis was chosen otherwise.

\subsection{SRILM}
\gls{srilm} is a toolkit for building and applying statistical language models, primarily for use in speech recognition, statistical tagging, and machine translation. It has been under development in the SRI Speech Technology and Research Laboratory since 1995. It consists of a set of C++ class libraries implementing language models, data stuctures and miscellaneous utility functions, a set of executable programs built on top of these libraries to perform standard tasks such as training and testing language models, and a collection of miscellaneous scripts facilitating minor related tasks \cite{stolcke2011srilm}. It mainly supports n-gram modelling, so it was used in the experimental part to train and evaluate word-based and class-based n-gram models. To automate certain tasks, the toolkit was accesed via a basic Python wrapper. SRILM was used under the SRI's Research Community License and the Python binding by Nathaniel Smith was used under the MIT Licence.
\subsection{RNNLM}
\section{Method}
Building language models can be roughly divided into two stages~--~first, the training data is gathered and pre-processed, and then the models are trained and evaluated. This section contains a description of the process conducted in the experimental part. The first subsection presents in detail the subsequent stages of the data preprocessing pipeline, while the second is devoted to the training process.
\subsection{Building the training and testing sets}
\label{subsection:trainingset}
The training data was extracted from the \gls{nkjp}, more precisely from the redistributable subcorpus and the one million manually annotated subcorpus. First, the files were divided into two categories~--~written texts and transcripts of speech. This process was automated using the metadata. The speech training corpus contains mostly the transcripted proceedings of the Polish parlament and investigation commisions, while the text corpus consists mainly of legal documents. The testing  set was built solely from parliamentary speeches. Testing the models on transcripts of spoken language seems reasonable if one wants to simulate the use of language models in an \gls{asr} system. However, the strong bias towards formal speech and writing is a serious issue. The training and testing corpora come from a specific domain and therefore cannot be treated a comprehensive representation of Polish (see Figures \ref{figure:commonunigrams}-\ref{figure:commontrigrams} presenting the most common n-grams from the training set). However, the one million subcorpus, although more representative, is far too small for the experiments to yield statistically significant results. Given this tradeoff, it seemed logical to sacrifice representativeness for the sake of meaningfullness.

\begin{table}[!htbp]
	\centering
	\caption{Contents of the text corpus.}
	\begin{tabular*}{.6\linewidth}{@{\extracolsep{\fill}}lr}
		source & word count \\
		\midrule
                books & 68 465 \\
                newspapers & 813 318 \\
                legal texts & 4 608 863 \\
                total  & 5 490 619 \\
	\end{tabular*}
\end{table}

\begin{table}[!htbp]
	\centering
	\caption{Contents of the speech corpus.}
	\begin{tabular*}{.6\linewidth}{@{\extracolsep{\fill}}lr}
		source & word count \\
		\midrule
                conversational speech  & 80 629 \\
                investigation commisions  & 4 198 129 \\
                parliamentary proceedings  & 13 895 902 \\
                total  & 18 174 660 \\
	\end{tabular*}
\end{table}

The XML files that constitute the \gls{nkjp} were parsed using a Python implementation of the ElementTree XML API. The XML files contain several layers of annotation, so it was possible to build four different types of corpora in a single run:

\begin{enumerate}
	\item plain -- plain text,
	\item lemma -- lemmatised text, 
	\item pos -- tags containing only information about the grammatical class (part of speech),
	\item gnc -- tags containing information about the grammatical class and selected grammatical categories (gender, number, case).
\end{enumerate}

Each corpus was split into speech and text. The twelve final training datasets are listed in Table \ref{table:datasets}. This nomenclature will be used consistently throughout the following chapters. 

\begin{table}[!htbp]
	\centering
	\caption{List of the training corpora.}
	\begin{tabular*}{.6\linewidth}{@{\extracolsep{\fill}}llll}
	\label{table:datasets}
		plain_full   & lemma_full   & pos_full   & gnc_full \\ 
		plain_text   & lemma_text   & pos_text   & gnc_text \\
		plain_speech & lemma_speech & pos_speech & gnc_speech \\ 
	\end{tabular*}
\end{table}

The content of the corpora was filtered during extraction. All corrupted and incomplete segments were discarded, all numerals expressed with digits were substituted with a common tag, and the text was converted to lowercase, stripped of punctuation, non-alphanumeric symbols and abbreviations. Moreover, only segments longer than three words were selected. 

\subsection{Training the n-gram models}
The n-gram models were built using the SRILM toolkit. First, the n-gram counts were computed for each corpus. Only unigrams, bigrams, and trigrams were taken under consideration. The results for the four types of corpora are presented in Tables \ref{table:countsplain}-\ref{table:countsgnc}. The counts include the start-of-sentence and end-of-sentence tokens. Note the evident data sparsity~--~in some cases, there are less trigrams present in the corpus

\begin{table}[!htbp]
	\centering
	\caption{Number of unique tokens in the plain corpus.}
	\begin{tabular*}{.6\linewidth}{@{\extracolsep{\fill}}l*3r}
	\label{table:countsplain}
		{}        & \multicolumn{1}{c}{text} & \multicolumn{1}{c}{speech} & \multicolumn{1}{c}{full}  \\
                unigrams  &  157 866   & 184 236    & 244 346\\
	        bigrams   &  1 540 010 & 4 083 052  & 5 111 189\\
		trigrams  &  4 669 098 & 14 549 205 & 12 843 843\\
	\end{tabular*}
\end{table}

\begin{table}[!htbp]
	\centering
	\caption{Number of unique tokens in the lemma corpus.}
	\begin{tabular*}{.6\linewidth}{@{\extracolsep{\fill}}l*3r}
	\label{table:countslemma}
		{}        & \multicolumn{1}{c}{text} & \multicolumn{1}{c}{speech} & \multicolumn{1}{c}{full}  \\
                unigrams  &  54 522    & 41 666    & 65 420     \\
	        bigrams   &  1 012 669 & 2 241 798 & 2 835 191  \\
		trigrams  &  2 555 053 & 8 314 572 & 10 380 493 \\
	\end{tabular*}
\end{table}

\begin{table}[!htbp]
	\centering
	\caption{Number of unique tokens in the pos corpus.}
	\begin{tabular*}{.6\linewidth}{@{\extracolsep{\fill}}l*3r}
	\label{table:countspos}
		{}        & \multicolumn{1}{c}{text} & \multicolumn{1}{c}{speech} & \multicolumn{1}{c}{full}  \\
                unigrams  &  34    & 34     & 34   \\
	        bigrams   &  851   & 905    & 935  \\
		trigrams  &  10238 & 14030  & 14689\\
	\end{tabular*}
\end{table}

\begin{table}[!htbp]
	\centering
	\caption{Number of unique tokens in the gnc corpus.}
	\begin{tabular*}{.6\linewidth}{@{\extracolsep{\fill}}l*3r}
	\label{table:countsgnc}
		{}        & \multicolumn{1}{c}{text} & \multicolumn{1}{c}{speech} & \multicolumn{1}{c}{full}  \\
                unigrams  & 462      & 411     & 475     \\
	        bigrams   & 34 874   & 38 107  & 45 689  \\
		trigrams  & 389 709  & 686 461 & 803 012 \\
	\end{tabular*}
\end{table}
The n-gram counts were then smoothed and interpolated. Several smoothing method were used

\begin{figure}[!htbp]
	  \centering
	  \includegraphics[height=10cm, width=15cm]{unigrams.png}
      \caption{Unigram distribution in the plain_full corpus.}
      \label{figure:unigramdistribution}
\end{figure}

\begin{figure}[!htbp]
	  \centering
	  \includegraphics[height=10cm, width=15cm]{bigrams.png}
      \caption{Bigram distribution in the plain_full corpus.}
      \label{figure:unigramdistribution}
\end{figure}

\begin{figure}[!htbp]
	  \centering
	  \includegraphics[height=10cm, width=15cm]{trigrams.png}
      \caption{Trigram distribution in the plain_full corpus.}
      \label{figure:unigramdistribution}
\end{figure}

\chapter{Results}
\label{chapter:results}
\section{Perplexity}
The perplexity values presented in Tables \ref{table:ppl_word}-\ref{table:ppl_gnc} were calculated on a test set of 20 000 utterances taken from the transcripted proceedings of the Polish parliament. The lowest perplexity achieved for a word n-gram model is 240.54. It should once again be noted that both the training and the test data come from a specific domain. This could significantly reduce the perplexity, as legal and parliamentary language is generally more predictable. For example, in \cite{bengio2003neural} it is reported that a Kneser-Ney smoothed back-off trigram model achieved a perplexity of 323 on the Brown Corpus (general English) and only 127 on the AP News Corpus. 

\begin{table}[!htbp]
	\centering
	\caption{Perplexity of the word text language models with Chen-Goodman's modified Kneser-Ney smoothing.}
	\label{table:ppl_word}
	\begin{tabular*}{.6\linewidth}{@{\extracolsep{\fill}}l*3r}
		{}        & \multicolumn{1}{c}{text} & \multicolumn{1}{c}{speech} & \multicolumn{1}{c}{full}  \\
		\midrule
                unigrams  & 4409.47  & 3510.78 & 3767.37\\
	        bigrams   & 1380.31  & 445.76  & 458.40\\
		trigrams  & 1231.23  & 240.54  & 242.31\\
	\end{tabular*}
\end{table}

\begin{table}[!htbp]
	\centering
	\caption{Perplexity of the lemma language models with Chen-Goodman's modified Kneser-Ney smoothing.}
	\label{table:ppl_lemma}
	\begin{tabular*}{.6\linewidth}{@{\extracolsep{\fill}}l*3r}
		{}        & \multicolumn{1}{c}{text} & \multicolumn{1}{c}{speech} & \multicolumn{1}{c}{full} \\
		\midrule
		unigrams  & 1844.97  & 1309.19 & 1654.56\\
	        bigrams   & 744.72   & 299.82  & 367.30\\
                trigrams  & 694.41   & 177.91  & 213.60\\
	\end{tabular*}
\end{table}

Because the models have different vocabulary sizes, the perplexity scores are not comparable across different types of models. For every type of models, the perplexity of speech models is lower than that of text models. This could be attributed to a larger training set, but in case of the word, lemma, and POS corpus, the perplexity of a model trained on the full corpus is consistently higher than that of a model trained on the speech corpus alone. This means that increasing the training set by incorporating the text data actually lead to a worse model, assuming that perplexity is a valid metric in this context. Another, rather unsuprising observation is the large and consistent decrease in perplexity with the n-gram order. Tables \ref{table:ppl_word} and \ref{table:ppl_lemma} show that the drop in perplexity is more pronounced in case of models with a larger vocabulary. Interestingly, the difference in perplexity between trigram and unigram models is much greater in case of models trained on speech data. For example, the perplexity of the word text trigram model is approximately 3.6 times lower than that of a word text unigram model, while the perplexity of a word speech trigram model is 14.6 times lower than that of its unigram counterpart.

\begin{table}[!htbp]
	\centering
	\caption{Perplexity of the POS language models with Witten-Bell smoothing.}
	\label{table:ppl_pos}
	\begin{tabular*}{.6\linewidth}{@{\extracolsep{\fill}}l*3r}
		{}        & \multicolumn{1}{c}{text} & \multicolumn{1}{c}{speech} & \multicolumn{1}{c}{full}  \\
		\midrule
		unigrams  & 11.25  & 10.40 & 10.37\\
	        bigrams   & 9.36   & 8.35  & 8.37\\
                trigrams  & 8.71   & 7.71  & 7.73\\
	\end{tabular*}
\end{table}

\begin{table}[!htbp]
	\centering
	\caption{Perplexity of the GNC language models with Good-Turing smoothing.}
	\label{table:ppl_gnc}
	\begin{tabular*}{.6\linewidth}{@{\extracolsep{\fill}}l*3r}
		{}        & \multicolumn{1}{c}{text} & \multicolumn{1}{c}{speech} & \multicolumn{1}{c}{full}  \\
		\midrule
		unigrams  & 90.04   & 82.16  & 81.38\\
	        bigrams   & 38.73   & 35.61  & 34.52\\
                trigrams  & 33.35   & 29.44  & 28.26\\
	\end{tabular*}
\end{table}

Tables \ref{table:ppl_pos} and \ref{table:ppl_gnc} present the perplexity results for the models based on morphosyntactic tags. The effect of the vocabulary size on perplexity is clearly visible. In case of the POS model, the perplexity of the model trained on the speech corpus is again lower than that of a model trained on the full corpus. However, the opposite is true for the GNC model. 

Generally, the perplexity scores hint that the models trained on speech transcripts are a better representation of the language for the purpose of \gls{asr}, as they tend to have a lower perplexity than models trained on written text, despite having a larger vocabulary. To further verify this hypothesis by eliminating the effect of the training set size, two equal sized training sets were built, one consisting of speech transcripts and the other of written texts. Then, four pairs of n-gram models were trained on these sets, this time ensuring that the vocabularies of corresponding models are identical. The perplexity of these models is presented in Tables \ref{table:ppl_word_small}-\ref{table:ppl_gnc_small}. The models based on speech data have a significantly lower perplexity for all four types of modeling units. The difference is more prominent in case of word and lemma models. For example, the perplexity of a speech-based word trigram model is more than four times lower than that of a text-based model. In case of POS and GNC models, the difference is still noticeable, although much less pronounced. Moreover, in case of speech models, perplexity still decreases more rapidly with the n-gram order. For example, using trigrams instead of unigrams results in a tenfold drop in perplexity of a speech-based model, and only a threefold decrease in case of a text-based model. 

\begin{table}[!htbp]
	\centering
	\caption{Comparison of perplexity of word text and speech models with equal vocabularies}
	\label{table:ppl_word_small}
	\begin{tabular*}{.4\linewidth}{@{\extracolsep{\fill}}l*2r}
		{}        & \multicolumn{1}{c}{text} & \multicolumn{1}{c}{speech}\\
		\midrule
		unigrams  & 3351.69   & 2287.98\\
	        bigrams   & 1051.36   & 367.79\\
                trigrams  & 942.74    & 226.06\\
	\end{tabular*}
\end{table}
\begin{table}[!htbp]
	\centering
	\caption{Comparison of perplexity of lemma text and speech models with equal vocabularies}
	\label{table:ppl_lemma_small}
	\begin{tabular*}{.4\linewidth}{@{\extracolsep{\fill}}l*2r}
		{}        & \multicolumn{1}{c}{text} & \multicolumn{1}{c}{speech}\\
		\midrule
		unigrams  & 1520.89  & 1149.99\\
	        bigrams   & 560.73   & 260.05\\
                trigrams  & 498.97   & 154.91\\
	\end{tabular*}
\end{table}
\begin{table}[!htbp]
	\centering
	\caption{Comparison of perplexity of POS text and speech models with equal vocabularies}
	\label{table:ppl_pos_small}
	\begin{tabular*}{.4\linewidth}{@{\extracolsep{\fill}}l*2r}
		{}        & \multicolumn{1}{c}{text} & \multicolumn{1}{c}{speech}\\
		\midrule
		unigrams  & 11.23  & 10.29\\
	        bigrams   & 9.27   & 8.23\\
                trigrams  & 8.62   & 7.56\\
	\end{tabular*}
\end{table}
\begin{table}[!htbp]
	\centering
	\caption{Comparison of perplexity of GNC text and speech models with equal vocabularies}
	\label{table:ppl_gnc_small}
	\begin{tabular*}{.4\linewidth}{@{\extracolsep{\fill}}l*2r}
		{}        & \multicolumn{1}{c}{text} & \multicolumn{1}{c}{speech}\\
		\midrule
		unigrams  & 159.14 & 156.54\\
	        bigrams   & 47.79  & 45.08\\
                trigrams  & 27.88  & 20.86\\
	\end{tabular*}
\end{table}

\begin{table}[!htbp]
	\centering
	\caption{Perplexity of neural models}
	\label{table:wer_neural}
	\begin{tabular*}{.4\linewidth}{@{\extracolsep{\fill}}lr}
		word   & 240.81\\
		lemma  & 216.64\\
		pos    & 7.12\\
		gnc    & 24.87\\
	\end{tabular*}
\end{table}

\section{Size}
\section{WERR}
The WERR values presented in this section were calculated using the n-best list rescoring framework described in Subsection \ref{subsection:wer}.

\label{section:werr}
\begin{figure}[!htbp]
	  \centering
	  \includegraphics[height=10cm, width=15cm]{word_werr_1.png}
	      \caption{Absolute word error reduction of the word model (testing set 1)}
	      \label{figure:word1}
\end{figure}

\begin{figure}[!htbp]
	  \centering
	  \includegraphics[height=10cm, width=15cm]{lemma_werr_1.png}
	      \caption{Absolute word error reduction of the lemma model (testing set 1)}
	      \label{figure:lemmy1}
\end{figure}

\begin{figure}[!htbp]
	  \centering
	  \includegraphics[height=10cm, width=15cm]{pos_werr_1.png}
	      \caption{Absolute word error reduction of the pos model (testing set 1)}
	      \label{figure:pos1}
\end{figure}

\begin{figure}[!htbp]
	  \centering
	  \includegraphics[height=10cm, width=15cm]{gnc_werr_1.png}
	      \caption{Absolute word error reduction of the gnc model (testing set 1)}
	      \label{figure:gnc1}
\end{figure}

\begin{figure}[!htbp]
	  \centering
	  \includegraphics[height=10cm, width=15cm]{word_werr_5.png}
	      \caption{Absolute word error reduction of the word model (testing set 5)}
	      \label{figure:word5}
\end{figure}

\begin{figure}[!htbp]
	  \centering
	  \includegraphics[height=10cm, width=15cm]{lemma_werr_5.png}
	      \caption{Absolute word error reduction of the lemma model (testing set 5)}
	      \label{figure:lemmy5}
\end{figure}

\begin{figure}[!htbp]
	  \centering
	  \includegraphics[height=10cm, width=15cm]{pos_werr_5.png}
	      \caption{Absolute word error reduction of the pos model (testing set 5)}
	      \label{figure:pos5}
\end{figure}

\begin{figure}[!htbp]
	  \centering
	  \includegraphics[height=10cm, width=15cm]{gnc_werr_5.png}
	      \caption{Absolute word error reduction of the gnc model (testing set 5)}
	      \label{figure:gnc5}
\end{figure}

\begin{table}[!htbp]
	\centering
	\caption{WER of n-gram models}
	\label{table:wer_ngram}
	\begin{tabular*}{.6\linewidth}{@{\extracolsep{\fill}}l*3r}
		{}        & \multicolumn{1}{c}{text} & \multicolumn{1}{c}{speech} & \multicolumn{1}{c}{full} \\
		\midrule
		word   & 12.15  & 8.67  & 8.55\\
		lemma  & 16.95  & 11.81 & 13.34\\
		pos    & 21.10  & 19.38 & 19.58\\
		gnc    & 12.66  & 12.57 & 12.25\\
	\end{tabular*}
\end{table}

\begin{table}[!htbp]
	\centering
	\caption{WER of neural models}
	\label{table:wer_neural}
	\begin{tabular*}{.4\linewidth}{@{\extracolsep{\fill}}lr}
		{}        &  \multicolumn{1}{c}{full} \\
		\midrule
		word  & 8.21\\
		lemma  & 12.55\\
		pos    & 17.65\\
		gnc    & 11.61\\
	\end{tabular*}
\end{table}

\chapter{Conclusions}
\label{chapter:conclusion}
The purpose of this thesis was to conduct a comparative analysis of morphosyntactic and semantic language models in the context of automatic speech recognition of Polish. Sixteen n-gram, class n-gram and neural network models were built and evaluated using several performance metrics: perplexity, size, and word error rate reduction on two n-best list rescoring tasks~--~one using the output of a real \gls{asr} system, and the other involving artificially generated recognition hypothesis.

Perplexity evaluation, although often criticized as a metric of the language model quality, revealed some trends which were further confirmed in the \gls{werr} experiments. First of all, there is a noticeable difference between language models trained on speech transcripts and those trained on written texts, regardless of the modelling unit used. The comparison of models with identical vocabularies showed that models trained on speech transcripts achieve lower perplexity than their text-based counterparts. Moreover, in case of words and lemmas, the speech-based trigram models outperform those trained on the full corporas.
Although more cumbersome, the \gls{werr} evaluation on a real \gls{asr} system is undoubtedly a more reliable evaluation method than perplexity. The results confirm what has been reported in \cite{dziadzio2015comparison}, this time on a much larger training set~--~n-gram models trained on speech transcripts indeed perform better than text-based models, even when morphosyntactic tags are used as the modelling units. Figure \ref{figure:total} presents a summary of results.

\begin{figure}[!htbp]
	  \centering
	  \includegraphics[height=10cm, width=15cm]{total_werr.png}
	  \caption[The absolute WERR achieved by respective models on n-best list rescoring]{The absolute WERR achieved by respective models on n-best list rescoring.}
	      \label{figure:total}
\end{figure}

The quality of the acoustic model may have been an important factor in the \gls{werr} calculation pipeline. Because a newer version of the recognition engine was unavailable, an alternative approach was proposed to simulate the operation of an advanced acoustic model. The mock n-best list rescoring is a compromise between proxy metrics like perplexity and using a real recognition system, as the generation of mock n-best lists is fully automated. It allows to simulate the rescoring task using any testing set without an actual \gls{asr} system. The results of the mock n-best list rescoring evaluation are presented in Figure \ref{figure:total_mock}. Note the very good score of the neural and \mbox{$n$-gram} GNC models, which both outperformed the word text model, despite a vocabulary order of magnitudes smaller.

\begin{figure}[!htbp]
	  \centering
	  \includegraphics[height=10cm, width=15cm]{total_simulated.png}
	  \caption[The absolute WERR achieved by respective models on mock n-best list rescoring]{The absolute WERR achieved by respective models on mock n-best list rescoring.}
	      \label{figure:total_mock}
\end{figure}

To sum up, the experimental part provided some valuable insights into how to optimise the language modelling of highly inflected language for speech recognition. First of all, the speech data once again proved to be a more valuable source of information than written texts \cite{dziadzio2015comparison}. Unfortunately, the conversation corpus in the \gls{nkjp} is too small to repeat the experiments on general spoken language. 

The second important observation is that a morphosyntactic model based on grammatical classes and three grammatical categories (gender, number, case) turned out to be a very effective way of modelling Polish for speech recognition. The GNC model performs similarly to a \mbox{lemma-based} model and does so at a fraction of the computational cost. It is likely that it is possible to find a similarly informative set of grammatical categories for other inflected languages. 

The last conclusion that can be drawn from the experimental part is that although the neural network models in most cases outperform their \mbox{$n$-gram} counterparts, the differences are rather slight. This can be attributed to numerous factors. Firstly, the neural models were trained on full corpora instead of just the speech transcript parts. Moreover, they were not optimised, as the values of parameters (the number of \gls{bptt} steps, the size of the hidden layer, the number of classes) were chosen so that the training part would be feasible on a consumer hardware. Additional experiments could definitely shed some light on the matter. It would be especially interesting to build and test a neural GNC model trained entirely on speech data, this time with a much larger hidden layer and a lot more \gls{bptt} steps, as it has the potential of being a viable alternative to \mbox{word-based} \mbox{$n$-grams}.



\bibliographystyle{acm}
\bibliography{bibliography}

\end{document}
