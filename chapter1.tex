\chapter{Introduction}
\label{chapter:intro}

\section{Automatic speech recognition}
\label{section:asr}
Automatic speech recognition (ASR) \nomenclature{ASR}{Automatic Speech Recognition} can be defined as independent, computer-driven transcription of spoken language into readable text in real time \cite{stuckless1994developments, jelinek1997statistical}. Although this process can be performed almost effortlessly by the human brain, it is extremely difficult to reverse-engineer\footnote{The observation that low-level sensorimotor skills require far more computational resources than high-level reasoning is known as the Moravec's paradox and has been formulated independently by several artificial intelligence researchers in the 1980s. As Moravec wrote: ``it is comparatively easy to make computers exhibit adult level performance on intelligence tests or playing checkers, and difficult or impossible to give them the skills of a one-year-old when it comes to perception and mobility''\cite{moravec1988mind}.}. Large-vocabulary continuous speech recognition falls into two distinct categories\nomenclature{LVCSR}{Large Vocabulary Continuous Speech Recognition}(LVCSR): speech transcription and speech understanding. The former aims to find the exact orthographic transcription of analysed utterance, while the latter aims to find its meaning. In this thesis we focus on speech transcription, because its performance can be reliably and unambiguously measured in terms of word recognition errors.
\subsection{Bayesian framework}
In general, the recogniser tries to determine the most likely word sequence $\hat{W}$ from possible hypotheses $W$ given a sequence of observed acoustic features $A$:
\begin{equation}
\label{equation:recogniser}
  \hat{W}=\max_{W}P(W|A)
\end{equation}
We can rearrange the conditional probability using the Bayes rule:
\begin{equation}
  \label{equation:bayesian}
  P(W|A)=\frac{P(A|W)P(W)}{P(A)}\propto{P(A|W)P(W)}
  \nomenclature{$P(W|A)$}{Probability of a word sequence $W$ being produced from acoustic evidence $A$}
\end{equation}
$P(W)$ is estimated by the language model (see section \ref{section:lm}), while $P(A|W)$ is computed using the acoustic model \cite{whittaker2000statistical}. Typical ASR systems use Hidden Markov Models (HMMs) \nomenclature{HMMs}{Hidden Markov Models} to model the sequential structure of speech signal \cite{juang1985mixture, baker1975dragon} and Gaussian Mixture Models (GMMs) \nomenclature{GMMs}{Gaussian Mixture Models} for modelling the emission distribution of HMMs \cite{mohamed2012acoustic, bourlard1994connectionist}.
\subsection{Feature extraction}
Speech is a non-stationary process, so to represent it as a succession of discrete stationary states, it is assumed that over a short period of time signal statistics do not differ from sample to sample. Under this assumption it is possible to extract statistically meaningful acoustic parameters (feature vectors) from a sampled speech waveform. Typically the features are calculated in 20-30 ms overlapping windows. The most popular method of spectral analysis is Linear Predictive Coding \nomenclature{LPC}{Linear Predictive Coding} and Mel-Frequency Cepstral Coefficients \nomenclature{MFCCs}{Mel-Frequency Cepstral Coefficients}. 
\begin{figure}[!ht]
  \label{figure:mfcc}
  \centering
    \begin{tikzpicture}[block/.style = {draw, rectangle, minimum width = 4cm, minimum height = 1cm, font = \small}]
      \node [block] (preemphasis) {Preemphasis};
      \node [block, right = of preemphasis] (framing) {Framing};
      \node [block, right = of framing] (windowing) {Windowing};
      \node [block, below = of windowing] (dft) {|(DFT)|\textsuperscript{2}};
      \node [block, below = of dft] (melfilter) {Mel Filters};
      \node [block, left = of melfilter] (dct) {DCT};
      \node [block, left = of dct] (delta) {Delta Coefficients};
      \path[draw,->]
      (preemphasis) edge (framing)
      (framing) edge (windowing)
      (windowing) edge (dft)
      (dft) edge (melfilter)
      (melfilter) edge (dct)
      (dct) edge (delta);
    \end{tikzpicture}
  \caption{Calculating MFCC}
\end{figure}
\subsection{Acoustic model}
The algorithm for calculating MFCCs is shown in figure \ref{figure:mfcc}. First, the speech waveform is subjected to high-frequency preemphasis in order to compensate for lip radiation and attenuation of high frequencies caused by the sampling process \cite{bhupinder2012preprocessing}. Typically, the signal is passed through a high-pass finite impulse response (FIR) \nomenclature{FIR}{Finite Impulse Response} filter:
\begin{equation}
H(z)=1-frac{a}{z}
\end{equation}
where $a$ usually takes values around $0.95$ \cite{nossair1995signal}. The signal is then divided into 

\section{Language models}
\label{section:lm}
Modeling language lies at the core of many natural language processing tasks, such as speech recognition and synthesis, document classification, grammar and spelling checking, parsing, machine translation, and information retrieval. In general, the task of statistical language models is to estimate the likelihood of a sequence of words. This information can then be used to perform binary classification (reject invalid sentences) or resolve ambiguities. In case of automatic speech recognition the language model guides and constrains the search among alternative word hypotheses \cite{glass2013automatic}.In a Bayesian framework presented in equation \ref{equation:bayesian}, the language model estimates the a priori likelihood by assigning probability $P(W)$ to each word sequence $W=\{w_{1}, \ldots, w_{n}\}$ such that $\sum_{W}P(W)=1$ \cite{rosenfeld2000two}. Since the search is usually performed unidirectionally, $P(W)$ can be formulated as a chain rule:
\begin{equation}
  P(W)=\prod^{n}_{i=1}P(w_{i}|h_{i})
  \nomenclature{$P(W)$}{Probability of a word sequence $W=\{w_{1}, \ldots, w_{n}\}$}
  \nomenclature{$P(w_{i}|h_{i})$}{Probability of a word $w_{i}$ given its history $h_{i}$}
\end{equation}
where $h_{i}=\{w_{1}, \ldots, w_{i-1}\}$ is the word history for $w_{i}$, often reduced to equivalence class $\phi(h_{i})$:
\begin{equation}
  P(w_{i}|h_{i})\approx P(w_{i}|\phi(h_{i}))
  \nomenclature{$\phi(h_{i})$}{Equivalence class of word history $h_{i}$}
\end{equation}
Good equivalence classes maximise information about the next word $w_{i}$ given its history, but also require a vast quantity of example sequences. The development of effective statistical language models is therefore limited by the availability of representative and machine readable text corpora.

\section{Polish vs. English}
\label{section:polish}

\section{Outline}
\label{section:outline}
