\chapter{Building the models}
\label{chapter:tools}
\section{Tools and resources}
This section describes the resources that served as a source of training data, and the tools that were used for gathering, filtering, and transforming the data, including XML-parsing, POS-tagging, lemmatization, and morphological disambiguation. It also presents two language modelling frameworks that enabled building large-scale n-gram and neural language models on consumer hardware.
\subsection{NKJP}
\label{section:nkjp}
Collecting high quality language data is a difficult task, as large and representative collections of modern texts are generally hard to obtain, if only for copyright reasons. One potential source of modern texts is obviously the Internet. However, web data can be extremely noisy and scraping, cleaning, and normalizing it would be a cumbersome process. It should also be considered that the language of the Internet is different from that used in speech. Indeed, transcripts of spoken language appear to be a more valuable source of training data in \gls{asr} applications, than written texts \cite{dziadzio2015comparison}. Moreover, using raw text data to build POS-based models would require a very accurate and robust morphological tagger. For these reasons, linguistic corpora have become essential in advanced language technology. Fortunately, there exists an extensive, publicly available reference corpus of Polish language.

The \gls{nkjp} is a shared initiative of four institutions: Institute of Computer Science at the Polish Academy of Sciences, Institute of Polish Language at the Polish Academy of Sciences, Polish Scientific Publishers PWN, and the Department of Computational and Corpus Linguistics at the University of Łódź. It has been carried out as a research-development project of the Ministry of Science and Higher Education. The full corpus contains over one and a half billion words. The list of sources for the corpora consists of literature, scientific journals, magazines, daily newspapers, and a variety of Internet texts. Most importantly, it contains transcripts of parliamentary proceedings and conversations. Moreover, the creators of the corpus claim that the conversations were chosen so that they represent both male and female speakers, in various age groups, coming from various regions in Poland \cite{lewandowska2012narodowy}. For that reason alone, \gls{nkjp} seemed like an excellent choice for the source of training data.

Another important feature of the \gls{nkjp} is that every text is accompanied by several layers of annotation. The metadata contain information about the text, such as title, author, and source, or~--~in case of speech transcripts~--~the topic of the conversation, the level of formality, background information about the speakers, and so on \cite{przepiorkowski2009xml}. More importantly, every lexical unit is described by several tags carrying information about its grammatical class and category. This enabled to easily filter the data by rejecting foreign words, incomplete or corrupted segments, punctuation, and non alpha-numeric characters. 

For the purpose of the experimental part of this thesis, the redistributable subcorpus was used as the main source of training data. It consists of all text of the full \gls{nkjp} that are free from intellectual property constraints. The sources include mostly texts of legal documents and transcripts of parliamentary and investigation commision proceedings. For detailed information about the contents of the final training set, see Section \ref{subsection:trainingset}.

\begin{table}[h!]
  \begin{center}
	  \caption{Contents of the redistributable subcorpus of the NKJP.}
	    \label{table:freenkjp}
	    \begin{tabular*}{.6\linewidth}{@{\extracolsep{\fill}}lr}
      source & word count \\
      \cmidrule{1-2}
      books & 67 000\\
      proceedings of investigation commisions & 4 623 000\\
      legal texts & 6 970 000\\
      parliamentary proceedings & 87 621 000\\
    \end{tabular*}
  \end{center}
\end{table}

\subsection{Concraft-pl}
Concraft-pl is a morphosyntactic tagger for Polish. It combines Maca -- a morphosyntactic segmentation and analysis tool and Concraft -- a morphosyntactic disambiguation library based on constrained conditional random fields \cite{waszczuk2012harnessing}. It was trained on the manually annotated subcorpus of the \gls{nkjp}, so that the tagsets are compatible. The tagger was used to tag plain text data in cases where the manual or semi-automatic annotation was unavailable. Although the output of the tagger is a list of most probable morphosyntactic tags with corresponding probabilities, only the one chosen by the disambiguation library was taken into consideration. The class mapping is therefore deterministic, which simplifies the languague model and enables to use existing \gls{nkjp} annotation when possible. As for lemmatization, Concraft-pl doesn't disambiguate between lemmas when the morphosyntactic tag is the same, so the NKJP annotation was used in almost every case and a random hypothesis was chosen otherwise.

\subsection{SRILM}
\gls{srilm} is a toolkit for building and applying statistical language models, primarily for use in speech recognition, statistical tagging, and machine translation. It has been under development in the SRI Speech Technology and Research Laboratory since 1995. It consists of a set of C++ class libraries implementing language models, data stuctures and miscellaneous utility functions, a set of executable programs built on top of these libraries to perform standard tasks such as training and testing language models, and a collection of miscellaneous scripts facilitating minor related tasks \cite{stolcke2011srilm}. It mainly supports n-gram modelling, so it was used in the experimental part to train and evaluate word-based and class-based n-gram models. To automate certain tasks, the toolkit was accesed via a basic Python wrapper. SRILM was used under the SRI's Research Community License and the Python binding by Nathaniel Smith was used under the MIT Licence.
\subsection{RNNLM}
\section{Method}
Building language models can be roughly divided into two stages~--~first, the training data is gathered and pre-processed, and then the models are trained and evaluated. This section contains a description of the process conducted in the experimental part. The first subsection presents in detail the subsequent stages of the data preprocessing pipeline, while the second is devoted to the training process.
\subsection{Building the training set}
\label{subsection:trainingset}
The training data was extracted from the \gls{nkjp}, more precisely from the redistributable subcorpus and the one million manually annotated subcorpus. First, the files were divided into two categories~--~written texts and transcripts of speech. This process was automated using the metadata. The speech corpus contains mostly transcripted proceedings of the Polish parlament and investigation commisions. The majority of the text corpus are legal documents. 
\begin{table}[!htbp]
	\centering
	\caption{Contents of the speech corpus.}
	\begin{tabular*}{.6\linewidth}{@{\extracolsep{\fill}}lr}
		source & word count \\
		\midrule
                conversational speech  & 80 629 \\
                investigation commisions  & 4 198 129 \\
                parliamentary proceedings  & 13 895 902 \\
                total  & 18 174 660 \\
	\end{tabular*}
\end{table}

\begin{table}[!htbp]
	\centering
	\caption{Contents of the text corpus.}
	\begin{tabular*}{.6\linewidth}{@{\extracolsep{\fill}}lr}
		source & word count \\
		\midrule
                books & 68 465 \\
                newspapers and articles & 813 318 \\
                legal texts & 4 608 863 \\
                total  & 5 490 619 \\
	\end{tabular*}
\end{table}

The XML files that constitute the \gls{nkjp} were parsed using a Python implementation of the ElementTree XML API. The XML files contain several layers of annotation, so it was possible to build four different types of corpora in a single run:

\begin{enumerate}
	\item plain -- plain text,
	\item lemma -- lemmatised text, 
	\item POS -- tags containing only information about the grammatical class (part of speech),
	\item GNC -- tags containing information about the grammatical class and selected grammatical categories (gender, number, case).
\end{enumerate}

Each corpus was split into speech and text, so there are eight final training datasets~--~plain_text, plain_speech, lemma_text, lemma_speech, pos_text, pos_speech, gnc_text, and gnc_speech. These names will be used in all tables and figures. The content of the corpus was filtered during extraction. Only segments longer than three words were selected. 

\begin{table}[!htbp]
	\centering
	\caption{Number of unique tokens in the plain corpus.}
	\begin{tabular*}{.6\linewidth}{@{\extracolsep{\fill}}l*3r}
		{}        & \multicolumn{1}{c}{text} & \multicolumn{1}{c}{speech} & \multicolumn{1}{c}{full}  \\
                unigrams  &  157 867   & 184 237   & 244 347\\
	        bigrams   &  1 540 010 & 4 083 052 & 5 111 189\\
		trigrams  &  639 039   & 1 480 416 & 2 119 127\\
	\end{tabular*}
\end{table}

\begin{table}[!htbp]
	\centering
	\caption{Number of unique tokens in the lemma corpus.}
	\begin{tabular*}{.6\linewidth}{@{\extracolsep{\fill}}l*3r}
		{}        & \multicolumn{1}{c}{text} & \multicolumn{1}{c}{speech} & \multicolumn{1}{c}{full}  \\
                unigrams  &  54 523    & 41 667    & 65 421 \\
	        bigrams   &  1 012 669 & 2 241 798 & 2 835 191 \\
		trigrams  &  679 041   & 1 609 414 & 2 248 516 \\
	\end{tabular*}
\end{table}

\begin{table}[!htbp]
	\centering
	\caption{Number of unique tokens in the pos corpus.}
	\begin{tabular*}{.6\linewidth}{@{\extracolsep{\fill}}l*3r}
		{}        & \multicolumn{1}{c}{text} & \multicolumn{1}{c}{speech} & \multicolumn{1}{c}{full}  \\
                unigrams  &  35   & 35     & 35   \\
	        bigrams   &  851  & 905    & 935  \\
		trigrams  &  8172 & 11956  & 12558\\
	\end{tabular*}
\end{table}

\begin{table}[!htbp]
	\centering
	\caption{Number of unique tokens in the gnc corpus.}
	\begin{tabular*}{.6\linewidth}{@{\extracolsep{\fill}}l*3r}
		{}        & \multicolumn{1}{c}{text} & \multicolumn{1}{c}{speech} & \multicolumn{1}{c}{full}  \\
                unigrams  & 463      & 412     & 475\\
	        bigrams   & 34 874   & 38 107  & 45 689\\
		trigrams  & 212 840  & 408 868 & 481 338\\
	\end{tabular*}
\end{table}

\subsection{Training the models}
