\documentclass[a4paper,10pt]{article}

\usepackage[english]{babel}
\usepackage[T1]{fontenc}
\usepackage[ansinew]{inputenc}
\usepackage{lmodern}
\usepackage{amsmath}
\usepackage{amsthm}
\usepackage{amsfonts}
\usepackage{tikz}
\usepackage{pgfplots}
\usetikzlibrary{calc}

\newcommand\gauss[2]{1/(#2*sqrt(2*pi))*exp(-((x-#1)^2)/(2*#2^2))}

\begin{document}

\tikzstyle{state}=[shape=circle,draw=black!50]
\tikzstyle{observation}=[shape=rectangle,draw=black!50]
\tikzstyle{lightedge}=[<-,dotted]
\tikzstyle{mainstate}=[state,thick]
\tikzstyle{mainedge}=[<-,thick]

\pgfplotsset{
  samples=50,
  domain=-3:2,
  no markers,
  axis x line*=bottom,
  axis y line*=left,
  axis line style=ultra thin,
  enlargelimits=upper,
  xtick=\empty,
  ytick=\empty
}


\begin{figure}[htbp]
  \begin{center}
    \begin{tikzpicture}[]
      % states
      \node[state] (s1) at (0,2) {$S_1$};
      \node[state] (s2) at (2,2) {$S_2$}
      edge [<-] node[auto,swap] {$a_{12}$} (s1)
      edge [loop above] node[auto] {$a_{22}$} ();
      \node[state] (s3) at (4,2) {$S_3$}
      edge [<-] node[auto, swap] {$a_{23}$} (s2)
      edge [loop above] node[auto] {$a_{33}$}();
      \node[state] (s4) at (6,2) {$S_4$}
      edge [<-] node[auto, swap] {$a_{34}$} (s3)
      edge [loop above] node[auto] {$a_{44}$}()
      edge [<-, bend left] node[auto] {$a_{24}$} (s2);
      \node[state] (s5) at (8,2) {$S_5$}
      edge [<-] node[auto, swap] {$a_{45}$} (s4);
      %gaussians
      \node(gauss1) at ($(s2)-(0, 2.5cm)$) {$b_{2}(y)$};
      \begin{axis} [height=2.5cm, width=3cm, at={($(s2)-(0, 2cm)$)}, very thin, anchor=south]
        \addplot [smooth] {0.4*\gauss{-1}{0.3}+0.2*\gauss{0}{0.6}+0.4\gauss{1}{0.4}};
      \end{axis}
      \node(gauss2) at ($(s3)-(0, 2.5cm)$) {$b_{3}(y)$};
      \begin{axis} [height=2.5cm, width=3cm, at={($(s3)-(0, 2cm)$)}, very thin, anchor=south]
        \addplot [smooth] {0.3*\gauss{-1}{0.2}+0.3*\gauss{0}{0.4}+0.4*\gauss{1}{0.2}};
      \end{axis}
      \node(gauss3) at ($(s4)-(0, 2.5cm)$) {$b_{4}(y)$};
      \begin{axis} [height=2.5cm, width=3cm, at={($(s4)-(0, 2cm)$)}, very thin, anchor=south]
        \addplot [smooth] {0.4*\gauss{-1}{0.4}+0.6*\gauss{1}{0.3}};
      \end{axis}
      % observations
      \node[observation] (y1) at (1,-2) {$y_1$}
      edge [lightedge] node[auto] {} (gauss1);
      \node[observation] (y2) at (2,-2) {$y_2$}
      edge [lightedge] node[auto] {} (gauss1);
      \node[observation] (y3) at (3,-2) {$y_3$}
      edge [lightedge] node[auto] {} (gauss2);
      \node[observation] (y4) at (4,-2) {$y_4$}
      edge [lightedge] node[auto] {} (gauss2);
      \node[observation] (y5) at (5,-2) {$y_5$}
      edge [lightedge] node[auto] {} (gauss3);
      \node[observation] (y6) at (6,-2) {$y_6$}
      edge [lightedge] node[auto] {} (gauss3);
      \node[observation] (y7) at (7,-2) {$y_7$}
      edge [lightedge] node[auto] {} (gauss3);
    \end{tikzpicture}
  \end{center}
  \caption{An HMM with 5 states An HMM with 4 states which can emit 2 discrete symbols $y_1$ or $y_2$.
    $a_{ij}$ is the probability to transition from state $s_i$ to state $s_j$.
    $b_j(y_k)$ is the probability to emit symbol $y_k$ in state $s_j$.
    In this particular HMM, states can only reach themselves or the adjacent state.}
\end{figure}
\end{document}
